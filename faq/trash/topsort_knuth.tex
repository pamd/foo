\newchapter{A Structured Program to Generate All\hfil\break
Topological Sorting Arrangements (1974)}%{with Jayme L. Szwarcfiter}

\def\chaptertitle{A Structured Program for Topological Sorting (1974)}

\begingroup

\def\K{\mathrel{:=}}
\def\C#1{{%\rightskip=0pt
\advance\ind by-2\hangindent\ind em\noindent\&{comment} #1\par\advance\ind by2}}
\def\rlab#1{#1}     % for labels that stand for sections of code: roman
\def\lab#1{\\{#1}}  % regular labels in italic

%Computer Science Dept., Stanford University, Stanford, Calif., 94305, USA
%
%and
%\author{Jayme L. Szwarcfiter}
%Universidade Federal do Rio de Janeiro, Argentina
%
%%\footnote
%{Research supported in part by the National Science
%Foundation, grant GJ 36473X, and by the Office of Naval Research, con-
%tract NR 044-402.}
%
%%\footnote 
%{Present address: Computing Laboratory, University of
%Newcastle upon Tyne, Claremont Tower, Newcastle upon Tyne, NE1 7RU,
%England.}
%
%%\footnote
%{Research supported by the Conselho Nacional de Pesquisas, Brazil.}
%
%Received 26 October 1973
%Revised version received 5 February 1974
%
%data structures programming languages combinatorial problems
%
\vskip12pt  %make bottom line closer to page number
\noindent
{\sl%
[This paper by Donald~E. ^{Knuth} and Jayme~L. ^{Szwarcfiter}, originally
published in {\sl Information Processing Letters} (April 1974), deals
with a problem that goes beyond the simple examples discussed in Chapter~2.
Although this problem is still ``academic'' and rather small by comparison
with a fullblown software system, it gives insight into the decomposition
of large tasks into comprehensible subtasks.]}
\bigskip

\noindent
An
%\footnote{}{\hangindent=0pt{}Research supported in part by the National
%Science Foundation, grant GJ~36473X; by the Office of Naval Research,
%contract NR~044-402, and by the Conselho Nacional de Pesquisas,
%Brazil.\par}
algorithm for topological sorting was presented by Knuth\cite{4} as an
example of typical interaction between linked and sequential forms of
data representation. The purpose of the present note is to extend the
algorithm so that it generates {\em all\/} solutions of the topological
sorting problem. The extended algorithm serves as an instructive
example of several important general issues related to backtracking,
procedures for changing recursion into iteration, manipulation of
^{data structures}, and the creation of well-structured programs.

Given a number \|n and a set of integer pairs $(i,j)$, where $1 \le
i$, $j \le n$, the problem of ^{topological sorting} is to find a
permutation $x_1x_2\ldots x_n$ of $\{1,2,\ldots,n\}$ such that \|i
appears to the left of \|j for all pairs $(i,j)$ that have been input.
It is convenient to denote input pairs by writing the relation
``$i\prec j$'' and saying ``\|i precedes \|j''. The topological
sorting problem is essentially equivalent to arranging the vertices of
a directed graph into a straight line so that all arcs go from left to
right. It is well known that such an arrangement is possible if and
only if there are no ^{oriented cycles} in the graph, i.e., if and only
if no relations of the form $$i_1\prec i_2,\quad i_2 \prec i_3,\quad
\ldots,\quad i_k \prec i_1$$ exist in the input, for any $\|k \ge 1$.
The problem in mathematical terms is to embed a given ^{partial order}
into a linear (total) order.

A natural way to solve this problem is to let $x_1$ be an element
having no predecessors, then to erase all relations of the form $x_1
\prec j$ and to let $x_2$ be an element $\ne x_1$ with no predecessors
in the system as it now exists, then to erase all relations of the
form $x_2 \prec j$, etc. It is not difficult to verify (cf.~[4]) that
this method will always succeed unless there is an oriented cycle in
the input. Moreover, in a sense it is the {\em only\/} way to proceed, since
$x_1$ must be an element without predecessors, and $x_2$ must be
without predecessors when all relations $x_1 \prec j$ are deleted, etc. This
observation leads naturally to an algorithm that finds {\em all\/} solutions
to the topological sorting problem; it is a typical example of a
``^{backtrack}'' procedure\cite{2,3}, where at every stage we consider a
subproblem of the form ``Find all ways to complete a given partial
permutation $x_1x_2\ldots x_k$ to a topological sort $x_1x_2\ldots
x_n$.'' The general method is to branch on all possible choices of
$x_{k+1}$.

A central problem in backtrack applications is to find a suitable way
to arrange the data so that it is easy to sequence through the
possible choices of $x_{k+1}$; in this case we need an efficient way to
discover the set of all elements $\ne \{x_1,\ldots,x_k\}$ which have no
predecessors other than $x_1,\ldots,x_k$, and to maintain this knowledge
efficiently as we move from one subproblem to another. It is not
satisfactory merely to make a brute-force search through all \|n 
possibilities for $x_{k+1}$, since this typically makes the program on the
order of \|n times slower than a method that avoids such searching.

The following procedure, written in an \ALGOL-like language using
^^{abstract data structures}``abstract'' data structures (cf.~^{Hoare}\cite{1}), shows how to solve the
problem with only order \|n additional units of storage. The procedure
is to be invoked by a main driver program of the form ``\rlab{read and
prepare the input\/}; \\{alltopsorts}(\O{0})''.
\smallskip

\Y\P\4\&{procedure}\1\ \37$\\{alltopsorts}(\|k)$;\5
\&{integer} \|k;\37
\&{value} \|k;\2\6
%\vskip2pt
\C{This procedure will output all topological sorting
arrangements beginning with a sequence $x_1\ldots x_k$ that has
already been output. Let $\|R = \{1,2, \ldots ,\|n\}\BS \{x_1, \ldots
,x_k\} $ be the set of all vertices not yet output; the procedure
assumes that, for all $y \in R$, the current value of global variable
\\{count}[\|y] is the number of relations $z \prec y$ for $z \in \|R$, and
that there is a linear list \|D containing precisely those elements $y
\in \|R$ such that $\\{count}[\|y] = 0$.  The execution of this
procedure may cause temporary changes to the contents of \|D and the
\\{count} array, but both will be restored to their entry values upon
exit;}\6\Y
\&{begin}\1\37
\&{integer} \|q, \37\\{base}; \6
\&{if} \|D not empty \1\&{then}\6
\&{begin}\1\37
\\{base}${}\K{}$rightmost element of \|D;\6
\1\&{repeat}\5
set \|q to rightmost element of \|D and delete it from \|D;\6
erase all relations of the form $q \prec j$;\6
output \|q in column $k+1$;\6
\&{if} $k = \|n - 1$ \1\&{then}\5
\\{newline};\2\6
\\{alltopsorts}${}(k+1){}$;\6
retrieve all relations of the form $q \prec j$;\6
insert \|q at the left of \|D;\6
\4\&{until} rightmost element of \|D = \\{base};\2\6
\4\&{end}\2\2\6
\4\&{end}.\2\6
%\fi
\Q

\noindent
The operations of erasing and retrieving relations will respectively
decrease and increase appropriate entries of the \\{count} array, and they
will also respectively insert and delete elements at the {\em right\/} of list
\|D.  

Thus, if we suppose that \|D contains $y_1 \ldots y_r$ on entry to
\\{alltopsorts}, for $\|r \ge 1$, the procedure will first set $\\{base}\K
\|y_r$, then $\|q\K \|y_r$.  Then it will decrease \\{count}[\|j] by
1 for each variable \|j such that $y_r\prec j$ was input; and if
$z_1,\ldots,z_s$ are the values of \|j whose count drops to zero at
this time, \|D will be changed to $y_1\ldots y_{r-1}z_1\ldots z_s$.
After outputting $y_r$, and doing \\{alltopsorts} beginning with
$x_1\ldots x_k y_r$, the procedure will restore each \\{count} to its initial
condition and will change \|D to $y_r y_1\ldots y_{r-1}$.  The same process will then
occur again with $q = y_{r-1}, y_{r-2} \ldots, y_1$, until finally all
sortings will have been produced and \|D will again be $y_1 \ldots y_r$; then
\\{alltopsorts} will exit. These remarks amount to a proof by
recursion induction that the algorithm is correct, since termination
is an obvious consequence of the fact that \|D contains at most $n-k$
entries.

Note that \|D operates as an {\em ^{output-restricted deque}\/},
since all deletions from \|D occur at its right and all insertions
occur at its ends. Therefore we can represent \|D as a list with one-way
linking^^{linking, one-way~vs. two way}. (It may be of interest to note that the authors' original
algorithm took \\{base} and \|q from the left of \|D while inserting $z_1 \ldots z_s$ 
and \|q at the right; this made \|D an input-restricted ^{deque}, so that
two-way linking was originally needed.  Thus, a slight perturbation of
the algorithm removed the need for one link.) Our program below uses
an array $\\{link}[\O{0}:n]$ to hold the pointers for \|D, which will be a
^{circular list}; a simple variable \|D points to the leftmost element, and
\\{link}[\|D] points to the rightmost.  

The erasure and retrieval operations are not difficult but they
require some comment. It is clear that a natural way to represent the
input relations for this purpose is to have a list for each \|i of all
\|j such that $i \prec j$. If there are \|m input relations in all,
entering in random order, we can handle this as in\cite{4} by having
integer arrays $\\{top}[1:\|n]$, $\\{suc}[1:\|m]$, $\\{next}[1:\|m]$, such
that $\\{top}[\|i]$ is the index \|p of the first relation for \|i, and
(if $\|p \neq 0$) $\\{suc}[\|p]$ is the corresponding \|j value and
$\\{next}[\|p]$ is the index of the next such relation. The erasing
operation is now easily programmed.

In order to convert the ^{recursion to iteration}, we will need a ^{stack}
for the local variables \|q and \\{base}. (It is not necessary to save
return addresses on the stack, since control always returns to one place
when $k > 0$ and to another when $k = 0$; furthermore it is unnecessary to
save \|k on the stack since it is easily updated across calls.) This
suggests that we introduce arrays $\|q[1:\|n]$ and $\\{base}[1:\|n]$.  However,
since the \\{count} entries are zero for all items \|q on the stack, it is
possible to save \|n locations by keeping an implicit ``\|q stack'' in the
\\{count} array; thus, a simple variable \|t contains the value of \|q at the
level just outside the current one, and \\{count}[\|t] holds the value on
the next level, etc.

Two more refinements will speed up the program.
First, we can test \|D for emptiness before actually calling the procedure,
thereby maintaining \|D as a nonempty list throughout the process; this
is a big advantage, since empty circular lists are always very
awkward. Second, we can realize that the procedure \\{alltopsorts} is
called most often when $k = \|n - 1$, and it can be greatly simplified in
that case.  

Putting these observations together yields the following
efficient machine-oriented program.

\Y\P\4\&{procedure}\37
\\{generate all topological sorting arrangements} $(\|m,\|n)$;\6
\&{integer} $\|m, \|n$;\5
\&{value} $\|m, \|n$;\5
\&{comment} \|m relations on \|n elements;\6
\4\&{begin}\37
\&{integer array} $\\{count}, \\{top}[1:\|n]$;\6
\&{integer array} $\\{suc}, \\{next}[1:\|m]$;\6
\&{integer array} $\\{link}, \\{base}[0:\|n]$;\6
\&{integer} $\|q, \|k, \|i, \|j, \|p, \|t, \|D, \\{D1}$;\6
\0\rlab{read and prepare the input}:\6
\&{for} $j\K 1$ \&{step} 1 \&{until} \|n \&{do}\37
$\\{count}[\|j]\K \\{top}[\|j]\K \O{0}$;\6
\&{for} $k\K 1$ \&{step} 1 \&{until} \|m \1\&{do}\6
\&{begin}\1\37
$\\{read}(i,j)$;\6
$\\{suc}[\|k] \K \|j$;\5
$\\{next}[\|k]\K \\{top}[\|i]$;\6
$\\{top}[\|i]\K \|k$;\5
$\\{count}[\|j]\K \\{count}[\|j]+1$;\6
\4\&{end};\2\2\6
$\\{link}[\O{0}]\K \|D \K \O{0}$;\6
\&{for} $\|j\K 1$ \&{step} 1 \&{until} \|n \1\&{do}\6
\&{begin}\1\37
\&{if} $\\{count}[\|j] = \O{0}$ \1\&{then}\6
\&{begin} $\\{link}[\|D] \K \|j$; $\|D \K \|j$ \&{end};\2\6 %%% SEMICOLON HERE? -no.
\4\&{end};\2\2\6
\&{if} $\|D = \O{0}$ \&{then}\37
\&{go to} \lab{done} \&{else} $\\{link}[\|D] \K \\{link}[\O{0}]$;\6
$\|k\K \O{0}$;\5
$\|t\K \O{0}$;\6
%
\0\lab{alltopsorts}:\37
\&{if} $\|k = \|n - 1$ \1\&{then}\5
\&{begin}\37
$\\{print}(\|D)$ \\{in column}: (\|n);\37
\\{newline}\5
\&{end}\6
\4\&{else}\37
\&{begin}\1\37
%
$\\{base}[\|k] \K \\{link}[\|D]$;\6
\1\&{repeat} \rlab{set \|q to rightmost and delete}:\6
$\|q\K \\{link}[\|D]$;\37
$\\{D1} \K \\{link}[\|q]$;\6
\0\rlab{erase relations beginning with \|q}:\6
$\|p\K \\{top}[\|q]$;\6
\&{while} $\|p \ne \O{0}$ \1\&{do}\6
\&{begin}\1\37
$\|j\K \\{suc}[\|p]$; \5
$\\{count}[\|j]\K \\{count}[\|j]-1$;\6
\&{if} $\\{count}[\|j] = \O{0}$ \1\&{then}\6
\&{begin} \&{if} $\|D = \|q$ \1\&{then} $\|D\K \|j$ \&{else} $\\{link}[\|j] \K \\{D1}$;\5
$\\{D1} \K \|j$;\6
\4\&{end};\2\6
$\|p\K \\{next}[\|p]$;\6
\4\&{end};\2\2\6
$\\{link} [\|D] \K \\{D1}$;\6
$\\{print}(q)$ \\{in column}: $(\|k+1)$;\6
\0\rlab{recursive call}:\6
\&{if} $\\{D1} = \|q$ \1\&{then}\6
\&{begin}\1\37
\&{comment} input contains an oriented cycle, and \|D~is empty;\6
\&{go to} \lab{done};\6
\4\&{end};\2\2\6
$\\{count}[\|q] \K \|t$; \5
$\|t\K \|q$;\6
$\|k\K \|k+1$; \5
\&{go to} \lab{alltopsorts};\6
\0\rlab{return when \|k positive}:\37
$\|k\K \|k-1$;\6
$\|q\K \|t$; \5
$\|t\K \\{count}[\|q]$; \5
$\\{count}[\|q] \K \O{0}$;\6
\0\rlab{retrieve relations beginning with \|q}:\6
$\|p\K \\{top}[\|q]$;\6
\&{while} $\|p \ne \O{0}$ \1\&{do}\6
\&{begin}\37
$\|j\K \\{suc}[\|p]$; \5
$\\{count}[\|j]\K \\{count}[\|j]+1$; \5
$p\K \\{next}[\|p]$;\6
\&{end};\2\6
\0\rlab{insert \|q at left}:\37
\&{comment} \\{link}[\|q] is still set properly;\6
$\\{link}[\|D] \K \|q$; \5
$\|D\K \|q$;\6
\4\&{until}\37
$\\{link}[\|D] = \\{base}[\|k]$;\2\6
\4\&{end};\2\2\6
\&{if} $\|k > \O{0}$ \&{then} \&{go to} \rlab{return when \|k positive};\6
\4\lab{done}: \&{end}.\2\6
\Q
%end code


The example input $$1\prec 3{,\quad} 2\prec 1{,\quad} 2\prec 4{,\quad} 4\prec 3{,\quad} 4\prec 5$$ will cause this program to print the five solutions

\medskip
\begingroup
\setbox1\hbox{\ $\to$\ }
\def\bx{\hbox to\wd1} 
\moveright5em\vbox{
\halign{#\ &\bx{\hfil#\hss}&\bx{\hfil#\hss}&\bx{\hfil#\hss}&\bx{\hfil#\hss}&\bx{\hfil#\hss}\cr
2&1&4&3&5 \cr
 & & &5&3 \cr
 &4&5&1&3 \cr
 & &1&3&5 \cr
 & & &5&3 \cr}}
\endgroup
\medbreak

\noindent
Notice that redundant printing has been suppressed. Alternatively, 
of course, the statement ``\\{print} (\|q) \\{in column}: $(\|k + 1)$''
could have been replaced by ``$\\{buffer}[\|k + 1] \K \|q$'', and ``\\{newline}''
replaced by ``$\\{printline}(\\{buffer}$)''.

This program is efficient in the sense that its inner loops are
reasonably fast, it uses only $O(m+n)$ words of memory, and it takes
at most $O(m + n)$ units of time per output. However, there may be
tremendous amounts of output, since the number of topological sortings
for large \|n is often very large. For example, when there are no
input relations at all, all $n!$ ^{permutations} are produced. (Note that
the number of digits printed in this worst case is not $n \cdot n!$,
but $n+n(n-1)+n(n-1)(n-2)+\cdots+n! = \lfloor n!\,e\rfloor-1$, about \|e per
permutation on the average.)

The volume of output and the running time can be reduced to 
$O(n\cdot2^n)$, if $O(2^n)$ more memory cells are allotted by
modifying the recursive procedure so that it ``remembers'' similar
situations.  We can add a new global variable corresponding to the
current value of the set $\{ x_1, \ldots, x_k\}$, and have the
procedure \\{alltopsorts} remember which sets it has seen before, and
where this occurred in the output. Whenever a set is repeated, the
output can now be replaced by a simple cross-reference to the
appropriate line. Thus, the above example would reduce to

%%%Should this be \ttverbatim??
\medskip
\begingroup
\setbox1\hbox{\ $\to$\ }
\def\bx{\hbox to\wd1} 
\moveright5em\vbox{
\halign{#\ &\bx{\hfil#\hss}&\bx{\hfil#\hss}&\bx{\hfil#\hss}&\bx{\hfil#\hss}&\bx{\hfil#\hss}\cr
1:&2&1&4&3&5\cr
2:& & & &5&3\cr
3:& &4&5&1&3\cr
4:& & &1&${}\to{}$&\ \hfilneg see line 1 \cr}}
\endgroup
\medbreak

\noindent
Adding ^^{memory and recursion}memory to recursive procedures is, of course, a standard way to
speed them up. Compare the exponential running time of

\Y
\&{integer} \&{procedure} \|F(\|n); \&{if} $n \le 1$ \&{then} \|n \&{else} $\|F(n - 1) + \|F(n - 2)$
\Y

%\Y\P\&{integer} \&{procedure} $\|F(\|n)$;\37
%\&{if} $n \le 1$ \1\&{then} \|n\2\5
%\&{else}\1\5
%$F(n - 1) + F(n - 2)$ \2
%\Q
%\Y

\noindent
to the linear running time of

%\Y
%\&{for} $n\K \O{0}$ \&{step} 1 \&{until} \|N \&{do} $F[\|n]\K {}$\&{if} $n \le 1$ \&{then} \|n \&{else} $\|F[n-1] + \|F[n-2]$.
%\Y

\Y\P\&{for} $n\K \O{0}$ \&{step} $1$ \&{until} \|N \1\&{do}\6
$\|F[\|n]\K {}$ \&{if} $\|n \le 1$ \&{then} \|n\5
\&{else}\37
$\|F[n-1] + \|F[n-2]$\2.
\Q
\Y

Our program contains the statement ``^{\&{go to}} \rlab{return when \|k
positive}'', which jumps {\em into\/} the \&{repeat} loop! Some people
consider this disgraceful; but the detailed discussion in\cite{5} [\see2]
points out that this \goto\ is not really harmful,
because it has been obtained by straightforward modification of an
original recursive program that was easily proved correct. The
complete documentation of a program should include its abstract form
and a discussion of the essentially mechanical transformations that
led to the final optimized form. It is possible to remove this \goto, 
but there would be no advantage in doing so. (See~[5] for
further examples and discussion.)

\newsection{References}

\vskip-1.5\belowsectionskip
\kern\parindent
\bibref{1}:^{Ole-Johan Dahl}, ^{Edsger W. Dijkstra}, and ^{C.~A.~R. Hoare},
 {\sl Structured programming\/}
(London: Academic Press, 1972), 220~pp.

\bibref{2}:^{Robert W. Floyd}, ``Nondeterministic algorithms,''
 {\sl Journal of the \ACM\/}\&{14} (1967), 636--644.

\bibref{3}:^{Solomon W. Golomb} and ^{Leonard D. Baumert}, ``Backtrack programming,''
 {\sl Journal of the~\ACM\/}~\&{12} (1965), 516--524.

\bibref{4}:^{Donald E. Knuth}, Fundamental algorithms: {\sl The Art of Computer
Programming\/}~\&{1}
(Reading, Massachusetts: Addison-Wesley, 1968), 634~pp.
Second edition, 1973.

\bibref{5}:Donald E. Knuth, ``Structured programming with \goto\ statements,''
 {\sl Computing Surveys\/}~\&{6} (December 1974), 261--301.
[Re\-printed with minor changes in {\sl Current Trends in Programming
Methodology}, ^{Raymond~T. Yeh} [Ed.], \&1 (Englewood Cliffs, New Jersey:
Prentice-Hall, 1977), 140--194;
 {\sl Classics in Software Engineering}, Edward Nash
 Yourdon [Ed.] (New York: Yourdon Press, 1979), 259--321.]

\bibref{6}:^{William M. Waite}, ``An efficient procedure for the generation
of closed subsets,'' {\sl Communications of the \ACM\/}~\&{10} (1967),
 169--171.

\vfill\eject

\endgroup

